\documentclass[a4paper,12pt,titlepage]{report}

\usepackage[T1]{fontenc}
\usepackage[french]{babel}
\usepackage[utf8]{inputenc}
\usepackage{amsmath}
\usepackage[english]{algorithm2e}
\usepackage{amssymb}
\usepackage{array}






\begin{document}

 \begin{titlepage}

\begin{center}




{ \huge \bfseries{Projet de Métaheuristique}}\\[0.4cm]


% Bottom of the page
\vfill
{\large Paul Beaujean\\ Clément Vincent}

\end{center}

\end{titlepage}
 \clearpage
\newpage

\tableofcontents \clearpage

\chapter{L'heuristique}

L'heuristique proposée ordonne les taches selon un poid. Plusieur méthode pour calculer le poid sont proposées :
\begin{itemize} \item $weight1 (agent,tache) = gain_{agent,tache}$

Ce poid permet d'assigner une tache à un agent en fesant en sorte que cela nous rapporte le plus possible.
\item $weight2 (agent,tache) = ocupation_{agent, tache}$

Ce poid permet d'assigner une tache à un agent en fesant en sorte que l'occupation soit minimale.
\item $weight3 (agent,tache) = \dfrac{ocupation_{agent,tache}}{capacity\_max_{agent}}$

Ce poid permet d'assigner une tache à un agent en fesant en sorte que l'occupation soit minimale, en considérant l'occupation absolue.
\item $weight4 (agent,tache) = \dfrac{-gain_{agent,tache}}{ocupation_{agent,tache}}$

Ce poid permet d'assigner une tache à un agent en considérant le ratio gain/occupation.
\end{itemize}

Le principe de l'heuristique est de considérer la différence entre la plus petite et la seconde plus petite valeure de poid pour chaque tache. Les taches sont assignées suivant l'ordre décroissant de cette différence.
Pour cela on calcule la désirabilité de chaque tache si on l'assigne à l'agent le plus intéressant.

$desirability_{task} = \max\limits_{agent}\min\limits_{autre\_agent}(weight(autre\_agent,task) - weight(agent,task))$

Ou encore :

$desirability_{task} = \min\limits_{autre\_agent}weight(autre\_agent,task) - weight(agent_{task},task)$

Avec :

$agent_{task} = \operatorname*{arg\,min}\limits_{agent}weight(agent,task)$

Seul les agents pour lesquels rajouter la tache ne fait pas dépasser leur capacité maximale sont considérés.



\fbox{\begin{algorithm}[H]
\KwData{$Task$ L'ensemble des taches.}

\Begin{
\While {$Task \neq \varnothing$}
{
	\For {$task \in Task$}
	{
		$Agent_{task} = \varnothing$
		\For {$agent \in Agent$}
		{
			\If{$ocupation_{agent,task} <= capacity\_max_{agent}$}
			{
				$Agent_{task} = Agent_{task} \cup \left\{agent\right\}$
			}
		}
		$agent_{task} = \operatorname*{arg\,min}\limits_{agent \in Agent_{task}}weight(agent,task)$
		$desirability_{task} = \min\limits_{autre\_agent \in Agent_{task}}weight(autre\_agent,task) - weight(agent_{task},task)$
	}
	$max\_desirability\_task = \operatorname*{arg\,max}\limits_{task \in Task}desirability_{task}$
	$assign(max\_desirability\_task)$
	$capacity\_max_{agent_{task}} = capacity\_max_{agent_{task}} - ocupation_{agent_{task},max\_desirability\_task}$
	$J = J\setminus\left\{max\_desirability\_task\right\}$

}

}


\end{algorithm}}


La fonction assign permet de remplir les variables d'affecations $x_{i,j}$ en la mettant à 1 pour la tache ayant la désirabilité maximum et l'agent qui va avec cette tache. Elle met cette variable à 0 pour toutes les autres taches associées avec ce même agent.
L'algorithme s'arrète quand toutes les taches ont été affectées. Si il n'y a plus d'agent ayant une capacité suffisante pour les taches restantes, l'algorithme ne trouve pas de solution admissible.



\chapter{Les voisinnages}

Etant donné une solution $x_{i,j}\in\left\{0,1\right\}$, on peut considérer les solutions x' correspondant à assigner une tache à un autre agent que celui auquel elle est assignée. On a $(nb\_agent-1)*nb\_tache$ solutions voisines. Toutes les solutions obtenuent ne seront pas admissibles, car un agent ayant déja atteint sa capacité maximale d'occupation se verra assigner une nouvelle tache.



Si on considère X la matrice des $x_{i,j}$, les X' obtenuent par permutions des lignes de X forment un voisinage. Cela revient à échanger toutes les taches effectuées par deux agents. Toutes les solutions obtenuent ne seront pas réalisables mais sauf dans des cas très particuliés, c'est a dire des instances dont il n'existe que très peu de solutions réalisables de base, il devrait y avoir des solutions réalisables dans ce voisinage.





Une autre idée de voisinage est de choisir une tache $j_{1}$, puis de chercher l'agent $i_{2}$ pour lequel le poid $weight(i_{2},j_{1})$ est minimum. Si l'agent $i_{2}$ est différent de celui auquel la tache est affectée $i_{1}$, chercher la tache $j_{2}$ parmis les taches effectuées par l'agent $i_{2}$ pour lequel le poid $weight(i_{1},j_{2})$ est minimum. Puis echanger $j_{1}$ et $j_{2}$.
Cela revient à échanger deux taches en prenant en compte la désirabilité pour choisir l'agent qui effectura la première tache, et pour choisir la seconde tache avec laquelle la première tache sera échangée.


\chapter{Exemples de voisinages}

On utilisera le deuxième ensemble de voisinage.

Voici l'instance utiliser pour les exemples de voisinage :
Nombre d'agent : 3\\
Nombre de tache : 6\\

Agent \#1\\
Capacité d'occupation maximal : 12\\
Gain par tache : 		5	7	7	6	8	5\\
Occupation par tache :	2	5	5	7	2	5

Agent \#2\\
Capacité d'occupation maximal : 11\\
Gain par tache : 		7	5	7	5	5	5\\
Occupation par tache :	5	2	7	7	3	3

Agent \#3\\
Capacité d'occupation maximal : 10\\
Gain par tache : 		5	6	5	7	7	5\\
Occupation par tache :	7	6	2	7	8	3

$ $

\section{Solution réalisable de départ}

$x = \begin{pmatrix}
 1&0&0&0&1&0 \\
 0&1&0&0&0&1 \\
 0&0&1&1&0&0 \\
\end{pmatrix}$\\
Occupation de chaque agent : 4 5 9\\
Valeur de la solution : $13 + 10 + 12 = 35$


$ $









\section{Premier ensemble de voisinage}

On enlève la tache1 à l'agent1 et on l'ajoute à l'agent2.
$x_{1} = \begin{pmatrix}
 0&0&0&0&1&0 \\
 1&1&0&0&0&1 \\
 0&0&1&1&0&0 \\
\end{pmatrix}$
Occupation de chaque agent : 2 10 9
Valeur de la solution : 37

On enlève la tache1 à l'agent1 et on l'ajoute à l'agent3.
$x_{2} = \begin{pmatrix}
 0&0&0&0&1&0 \\
 0&1&0&0&0&1 \\
 1&0&1&1&0&0 \\
\end{pmatrix}$\\
Occupation de chaque agent : 2 5 16
Solution non admissible

On enlève la tache2 à l'agent2 et on l'ajoute à l'agent1.
$x_{3} = \begin{pmatrix}
 1&1&0&0&1&0 \\
 0&0&0&0&0&1 \\
 0&0&1&1&0&0 \\
\end{pmatrix}$
Occupation de chaque agent : 9 3 9
Valeur de la solution : 37

On enlève la tache2 à l'agent2 et on l'ajoute à l'agent3.
$x_{4} = \begin{pmatrix}
 1&0&0&0&1&0 \\
 0&0&0&0&0&1 \\
 0&1&1&1&0&0 \\
\end{pmatrix}$
Occupation de chaque agent : 4 3 15
Solution non admissible

On enlève la tache3 à l'agent3 et on l'ajoute à l'agent1.
$x_{5} = \begin{pmatrix}
 1&0&1&0&1&0 \\
 0&1&0&0&0&1 \\
 0&0&0&1&0&0 \\
\end{pmatrix}$
Occupation de chaque agent : 9 5 7
Valeur de la solution : 37

On enlève la tache3 à l'agent3 et on l'ajoute à l'agent2.
$x_{6} = \begin{pmatrix}
 1&0&0&0&1&0 \\
 0&1&1&0&0&1 \\
 0&0&0&1&0&0 \\
\end{pmatrix}$
Occupation de chaque agent : 4 12 7
Solution non admissible

On enlève la tache4 à l'agent3 et on l'ajoute à l'agent1.
$x_{7} = \begin{pmatrix}
 1&0&0&1&1&0 \\
 0&1&0&0&0&1 \\
 0&0&1&0&0&0 \\
\end{pmatrix}$
Occupation de chaque agent : 11 5 2
Valeur de la solution : 34

On enlève la tache4 à l'agent3 et on l'ajoute à l'agent2.
$x_{8} = \begin{pmatrix}
 1&0&0&0&1&0 \\
 0&1&0&1&0&1 \\
 0&0&1&0&0&0 \\
\end{pmatrix}$
Occupation de chaque agent : 4 12 2
Solution non admissible

On enlève la tache5 à l'agent1 et on l'ajoute à l'agent2.
$x_{9} = \begin{pmatrix}
 1&0&0&0&0&0 \\
 0&1&0&0&1&1 \\
 0&0&1&1&0&0 \\
\end{pmatrix}$
Occupation de chaque agent : 2 8 9
Valeur de la solution : 32

On enlève la tache5 à l'agent1 et on l'ajoute à l'agent3.
$x_{10} = \begin{pmatrix}
 1&0&0&0&0&0 \\
 0&1&0&0&0&1 \\
 0&0&1&1&1&0 \\
\end{pmatrix}$
Occupation de chaque agent : 2 5 17
Solution non admissible

On enlève la tache6 à l'agent2 et on l'ajoute à l'agent1.
$x_{11} = \begin{pmatrix}
 1&0&0&0&1&1 \\
 0&1&0&0&0&0 \\
 0&0&1&1&0&0 \\
\end{pmatrix}$
Occupation de chaque agent : 9 2 9
Valeur de la solution : 35

On enlève la tache6 à l'agent2 et on l'ajoute à l'agent3.
$x_{12} = \begin{pmatrix}
 1&0&0&0&1&0 \\
 0&1&0&0&0&0 \\
 0&0&1&1&0&1 \\
\end{pmatrix}$
Occupation de chaque agent : 4 2 12
Solution non admissible

\section{Deuxième ensemble de voisinage}

$x_{1}$ est la solution issue de la permutation entre la première ligne et la deuxième ligne de la matrice x. Les taches effectuées par l'agent1 sont maintenant effectuées par l'agent2, et les taches effectuées par l'agent2 sont maintenant effectuées par l'agent1.
$x_{1} = \begin{pmatrix}
 0&1&0&0&0&1 \\
 1&0&0&0&1&0 \\
 0&0&1&1&0&0 \\
\end{pmatrix}$\\
Occupation de chaque agent : 10 8 9
Valeur de la solution : 36

$x_{2}$ est la solution issue de la permutation entre la première ligne et la troisième ligne de la matrice x. Les taches effectuées par l'agent1 sont maintenant effectuées par l'agent3, et les taches effectuées par l'agent3 sont maintenant effectuées par l'agent1.
$x_{2} = \begin{pmatrix}
 0&0&1&1&0&0 \\
 0&1&0&0&0&1 \\
 1&0&0&0&1&0 \\
\end{pmatrix}$\\
Occupation de chaque agent : 12 5 15
Solution non admissible

$x_{3}$ est la solution issue de la permutation entre la deuxième ligne et la troisième ligne de la matrice x. Les taches effectuées par l'agent2 sont maintenant effectuées par l'agent3, et les taches effectuées par l'agent3 sont maintenant effectuées par l'agent2.
$x_{3} = \begin{pmatrix}
 1&0&0&0&1&0 \\
 0&0&1&1&0&0 \\
 0&1&0&0&0&1 \\
\end{pmatrix}$\\
Occupation de chaque agent : 4 14 9
Solution non admissible

\section{Troisième ensemble de voisinage}

On prendra le gain comme critère pour l'exemple

Pour la première tache, l'agent2 a un meilleur gain que l'agent1. La tache 1 va donc etre affectée à l'agent2. Parmis les taches de l'agent2, celle qui permet le plus de gain pour l'agent1 et la tache 2. Elle va donc etre affectée à l'agent1.
$x_{1} = \begin{pmatrix}
 0&1&0&0&1&0 \\
 1&0&0&0&0&1 \\
 0&0&1&1&0&0 \\
\end{pmatrix}$
Occupation de chaque agent : 7 8 9
Valeur de la solution : 39

Pour la deuxième tache, l'agent1 a un meilleur gain que l'agent2. La tache 2 va donc etre affectée à l'agent 1. Parmis les taches de l'agent1, celle qui permet le plus de gain pour l'agent2 et la tache 1. Elle va donc etre affectée à l'agent2.
On obtient la même solution que précedement.

Pour la troisième tache, l'agent1 a un meilleur gain que l'agent3. La tache 3 va donc etre affectée à l'agent1. Parmis les taches de l'agent1, celle qui permet le plus de gain pour l'agent3 et la tache 5. Elle va donc etre affectée à l'agent3.
$x_{2} = \begin{pmatrix}
 1&0&1&0&0&0 \\
 0&1&0&0&0&1 \\
 0&0&0&1&1&0 \\
\end{pmatrix}$
Occupation de chaque agent : 7 5 15
Solution non admissible

Pour les taches restantes, les agents auquels elles ont été affecté ont le meilleur gain. Il n'y a donc pas de changement à faire.

\chapter{Solutions obtenues}

Pour chaque instance, on donne la valeure de la solution obtenue arpès l'heuristique, puis après la monté, pour le poid3 et le poid4.
Un X veux dire que l'heuristique ne trouve pas de solution réalisable, et un - signifie que la valeure n'a pas pu etre récupérée.

\section{gap1}

\begin{tabular}{|l|l|l|l|l|l|l|l|}
  \hline
  gap1 & nb\_it & poid3 & montée & nb\_it & poid4 & montée & opt \\
  \hline
  I-515-1 & 14 & X & X & 15 & 309 & 312 & 336  \\
   I-515-2 & 14 & X & X & 14 & X & X & 327  \\
 I-515-3 & 15 & 278 & 303 & 15 & 286 & 317 & 339 \\
 I-515-4 & 15 & 297 & 319 & 14 & X & X & 341  \\
 I-515-5 & 15 & 302 & 317 & 15 & 297 & 317 & 326  \\
  \hline
\end{tabular}

\section{gap2}

\begin{tabular}{|l|l|l|l|l|l|l|l|}
  \hline
  gap2 & nb\_it & poid3 & montée & nb\_it & poid4 & montée & opt \\
  \hline
  I-520-1 & 20 & 341 & 419 & 20 & 378 & 413 & 434  \\
  I-520-2 & 19 & X & X & 19 & X & X & 436  \\
  I-520-3 & 20 & 335 & 416 & 20 & 359 & 403 & 420  \\
  I-520-4 & 20 & 358 & 396 & 20 & 348 & 348 & 419  \\
  I-520-5 & 20 & 365 & 415 & 20 & 359 & 415 & 428  \\
  \hline
\end{tabular}

\section{gap3}

\begin{tabular}{|l|l|l|l|l|l|l|l|}
  \hline
  gap3 & nb\_it & poid3 & montée & nb\_it & poid4 & montée & opt \\
  \hline
  I-525-1 & 25 & 509 & 567 & 25 & 519 & 567 & 580  \\
  I-525-2 & 25 & 483 & 536 & 25 & 489 & 532 & 564  \\
  I-525-3 & 25 & 516 & 563 & 25 & 528 & 568 & 573  \\
  I-525-4 & 25 & 496 & 538 & 25 & 528 & 532 & 570  \\
  I-525-5 & 25 & 482 & 549 & 25 & 497 & 554 & 564  \\
  \hline
\end{tabular}

\section{gap4}

\begin{tabular}{|l|l|l|l|l|l|l|l|}
  \hline
  gap4 & nb\_it & poid3 & montée & nb\_it & poid4 & montée & opt \\
  \hline
  I-530-1 & 30 & 528 & 635 & 30 & 572 & 639 & 656  \\
  I-530-2 & 30 & 523 & 601 & 30 & 582 & 603 & 644  \\
  I-530-3 & 30 & 559 & 651 & 30 & 600 & 627 & 673  \\
  I-530-4 & 30 & 544 & 621 & 30 & 552 & 619 & 647  \\
  I-530-5 & 30 & 556 & 647 & 30 & 585 & 647 & 664  \\
  \hline
\end{tabular}

\section{gap5}

\begin{tabular}{|l|l|l|l|l|l|l|l|}
  \hline
  gap5 & nb\_it & poid3 & montée & nb\_it & poid4 & montée & opt \\
  \hline
  I-824-1 & 24 & 478 & 556 & 24 & 486 & 554 & 563  \\
  I-824-2 & 24 & 450 & 529 & 24 & 481 & 534 & 558  \\
  I-824-3 & 24 & 458 & 545 & 24 & 470 & 547 & 564  \\
  I-824-4 & 24 & 461 & 562 & 24 & 475 & 552 & 568  \\
  I-824-5 & 24 & 479 & 541 & 24 & 471 & 528 & 559  \\
  \hline
\end{tabular}

\section{gap6}

\begin{tabular}{|l|l|l|l|l|l|l|l|}
  \hline
  gap6 & nb\_it & poid3 & montée & nb\_it & poid4 & montée & opt \\
  \hline
  I-832-1 & 32 & 637 & 753 & 32 & 659 & 748 & 761  \\
  I-832-2 & 32 & 640 & 739 & 32 & 677 & 748 & 759  \\
  I-832-3 & 32 & 629 & 728 & 32 & 635 & 732 & 758  \\
  I-832-4 & 32 & 640 & 735 & 32 & 630 & 713 & 752  \\
  I-832-5 & 32 & 638 & 720 & 32 & 631 & 741 & 747  \\
  \hline
\end{tabular}

\section{gap7}

\begin{tabular}{|l|l|l|l|l|l|l|l|}
  \hline
  gap7 & nb\_it & poid3 & montée & nb\_it & poid4 & montée & opt \\
  \hline
  I-840-1 & 40 & 797 & 927 & 40 & 795 & 913 & 942  \\
  I-840-2 & 40 & 820 & 938 & 40 & 823 & 888 & 949  \\
  I-840-3 & 40 & 815 & 954 & 40 & 848 & 951 & 968  \\
  I-840-4 & 40 & 801 & 914 & 40 & 801 & 900 & 945  \\
  I-840-5 & 40 & 793 & 950 & 40 & 844 & 925 & 961  \\
  \hline
\end{tabular}

\section{gap8}

\begin{tabular}{|l|l|l|l|l|l|l|l|}
  \hline
  gap8 & nb\_it & poid3 & montée & nb\_it & poid4 & montée & opt \\
  \hline
  I-848-1 & 48 & 978 & 1104 & 48 & 978 & 1092 & 1133  \\
  I-848-2 & 48 & 964 & 1100 & 48 & 977 & 1064 & 1134  \\
  I-848-3 & 48 & 964 & 1110 & 48 & 994 & 1106 & 1141  \\
  I-848-4 & 48 & 958 & 1049 & 48 & 980 & 1063 & 1117  \\
  I-848-5 & 48 & 963 & 1074 & 48 & 1018 & 1064 & 1127  \\
  \hline
\end{tabular}

\section{gap9}

\begin{tabular}{|l|l|l|l|l|l|l|l|}
  \hline
  gap9 & nb\_it & poid3 & montée & nb\_it & poid4 & montée & opt \\
  \hline
  I-1030-1 & 30 & 577 & 684 & 30 & 606 & 675 & 709  \\
  I-1030-2 & 30 & 590 & 693 & 30 & 611 & 688 & 717  \\
  I-1030-3 & 30 & 629 & 684 & 30 & 594 & 675 & 712  \\
  I-1030-4 & 30 & 638 & 709 & 30 & 631 & 697 & 723  \\
  I-1030-5 & 30 & 603 & 684 & 30 & 592 & 668 & 706  \\
  \hline
\end{tabular}

\section{gap10}

\begin{tabular}{|l|l|l|l|l|l|l|l|}
  \hline
  gap10 & nb\_it & poid3 & montée & nb\_it & poid4 & montée & opt \\
  \hline
  I-1040-1 & 26 & 809 & 945 & 28 & 841 & 944 & 958  \\
  I-1040-2 & 19 & 837 & 936 & 19 & 803 & 924 & 963  \\
  I-1040-3 & 21 & 827 & 932 & 14 & 847 & 913 & 960  \\
  I-1040-4 & 24 & 770 & 921 & 17 & 808 & 910 & 947  \\
  I-1040-5 & 21 & 806 & 923 & 23 & 808 & 921 & 947  \\
  \hline
\end{tabular}

\section{gap11}

\begin{tabular}{|l|l|l|l|l|l|l|l|}
  \hline
  gap11 & nb\_it & poid3 & montée & nb\_it & poid4 & montée & opt \\
  \hline
  I-1050-1 & 40 & 809 & 945 & 28 & 841 & 944 & 1139  \\
  I-1050-1 & 40 & 809 & 945 & 28 & 841 & 944 & 1178  \\
  I-1050-1 & 40 & 809 & 945 & 28 & 841 & 944 & 1195  \\
  I-1050-1 & 40 & 809 & 945 & 28 & 841 & 944 & 1171  \\
  I-1050-1 & 40 & 809 & 945 & 28 & 841 & 944 & 1171  \\
  \hline
\end{tabular}

\section{gap12}

\begin{tabular}{|l|l|l|l|l|l|l|l|}
  \hline
  gap12 & nb\_it & poid3 & montée & nb\_it & poid4 & montée & opt \\
  \hline
  I-1060-1 & 40 & 809 & 945 & 28 & 841 & 944 & 1451  \\
  I-1060-1 & 40 & 809 & 945 & 28 & 841 & 944 & 1449  \\
  I-1060-1 & 40 & 809 & 945 & 28 & 841 & 944 & 1433  \\
  I-1060-1 & 40 & 809 & 945 & 28 & 841 & 944 & 1447  \\
  I-1060-1 & 40 & 809 & 945 & 28 & 841 & 944 & 1446  \\
  \hline
\end{tabular}

\end{document}

